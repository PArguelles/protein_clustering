%!TEX root = ../template.tex
%%%%%%%%%%%%%%%%%%%%%%%%%%%%%%%%%%%%%%%%%%%%%%%%%%%%%%%%%%%%%%%%%%%%
%% abstrac-en.tex
%% NOVA thesis document file
%%
%% Abstract in English
%%%%%%%%%%%%%%%%%%%%%%%%%%%%%%%%%%%%%%%%%%%%%%%%%%%%%%%%%%%%%%%%%%%%


Proteins are very complex and important molecules that carry out a wide range of functions essential to life. The role of any given protein within a cell is heavily determined by its structure, which is recognized as a valuable resource of information when studying proteins. In applications such protein docking or phylogenetics, there is the need to compare such structures in order to obtain relevant information about the proteins that are being considered. As such, there is also a need to specify a some kind of measure that is able to indicate if two protein structures are similar or not. Currently, there are a few different measures that can be used for this task, however, due to the inherent complexity of protein structures it is very hard for a measure to take into account the numerous possible variations and perfectly quantify the dissimilarity among them. 

Considering this issue, with this work we aim to use multiple clustering algorithms and experiment with different structure similarity measures, in an attempt to find effective ways of grouping protein structures with the goal of obtain useful information that can be used in the previously mentioned applications.

\begin{keywords}
Proteins, Machine learning, Unsupervised learning, Clustering, Protein structures, Structural similarity measures
\end{keywords} 
