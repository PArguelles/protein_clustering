%!TEX root = ../template.tex
%%%%%%%%%%%%%%%%%%%%%%%%%%%%%%%%%%%%%%%%%%%%%%%%%%%%%%%%%%%%%%%%%%%%
%% abstrac-pt.tex
%% NOVA thesis document file
%%
%% Abstract in Portuguese
%%%%%%%%%%%%%%%%%%%%%%%%%%%%%%%%%%%%%%%%%%%%%%%%%%%%%%%%%%%%%%%%%%%%

As proteínas são moléculas de alta importância e complexidade que desempenham uma grande diversidade de funções essenciais à vida. O papel de uma dada proteína dentro de uma célula é fortemente influenciado pela sua estrutura, que é reconhecida como um valioso recurso de informação no estudo de proteínas. Em aplicações como o \textit{docking} ou a análise filogenética de proteínas, há uma necessidade de comparar tais estruturas de forma a obter informação relevante sobre as proteínas que estamos a considerar. Como tal, também há a necessidade de especificar uma medida que seja capaz de indicar se duas estruturas de proteínas são ou não semelhantes. Atualmente, há medidas diferentes que podem ser usadas para esta tarefa, no entanto, devido à inerente complexidade das estruturas de proteínas é muito difícil para uma medida ter em conta as numerosas variações possíveis e quantificar perfeitamente as diferenças entre elas.  

Tendo estes problemas em consideração, neste trabalho vamos usar múltiplos algoritmos de \textit{clustering} e experimentar diferentes medidas de semelhança, numa tentativa de encontrar maneiras efetivas de agrupar estruturas de proteínas com o objectivo de obter informação útil, que possa ser usada nas aplicações mencionadas anteriormente.

% Palavras-chave do resumo em Português
\begin{keywords}
Proteínas, Aprendizagem automática, Aprendizagem não-supervisionada, Clustering, Medidas de semelhança estrutural
\end{keywords}
% to add an extra black line
